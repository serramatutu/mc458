\documentclass[12pt,a4paper]{article}

\usepackage[brazil]{babel}
\usepackage{amsmath}
\usepackage{scrextend}

\title{Projeto de Algoritmo com Implementação nº 2}
\date{20 de dezembro de 2020}
\author{Lucas Valente Viegas de Oliveira Paes - RA 220958}

\usepackage{etoolbox}
\AtBeginEnvironment{quote}{\itshape}

\begin{document}
  \maketitle

  \section{Prova de subestrutura ótima}
  Suponha que $C$ é um caminho ótimo até determinado bloco $b_n$, sendo $n$ o número blocos desde o chão até $b_n$, incluindo $b_n$. Considere $|C|$ o custo de um caminho e $|b|$ o custo de um bloco. Teremos, então, dois casos:
  \begin{labeling}{$n = 1$}
    \item [$n = 1$] Nesse caso, $b_n$ está no chão, na primeira linha da matriz de escalada. Assim, simplesmente, $|C| = |b_n|$.
    \item [$n > 1$] Nesse caso, há pelo menos um bloco $b_i$, com $i < n$, além de $b_n$ em $C$, de modo que este possa ser decomposto em pelo menos dois caminhos: $C_1$, do chão até $b_i$, e $C_2$, de $b_i$ até $b_n$. Para representar essa situação, diremos que $C = C_1 + C_2$.
  \end{labeling}

  No primeiro caso, se $C$ não fosse caminho ótimo, teríamos uma contradição trivial. Logo, $C$ é caminho ótimo.

  No segundo caso, suponha que $C_1$ não é caminho ótimo. Então, existe $C_0$ de forma que $|C_0| < |C_1|$. Assim, existe $C' = C_0 + C_2$ de forma que $|C'| = |C_0| + |C_2| < |C_1| + |C_2| = |C|$. Como isso é uma contradição, então $C_1$ é um caminho ótimo. Com raciocínio análogo, prova-se que $C_2$ também é caminho ótimo.

  Ainda no segundo caso, vale a pena observar que estamos apenas subindo na parede de escalada, passando somente uma vez por cada um de seus níveis. Assim, é impossível que um bloco $b_i$ se repita em um mesmo caminho $C$. Portanto, quaisquer caminhos $C_1$ e $C_2$ tal que $C = C_1 + C_2$ são independentes, ou seja, não compartilham blocos.

  Eis a prova de subestrutura ótima do problema da escalada, além da independência de seus subproblemas.

  \section{Recorrência do problema}
  Seja $C(b_{i,j})$ termo que representa o menor custo para se ir do chão até um bloco $b_{i,j}$ localizado na $i$-ésima linha e $j$-ésima coluna da matriz de largura $m$. Além disso, seja $|b_{i,j}|$ o custo desse bloco. Então,

  \begin{align*}\label{recorrencia}
    C(b_{i,j}) = 
    \begin{cases}
        |b_{i,j}|,& \text{se } i = 1\\
      \infty,& \text{se } j \notin [1,m] \\
      min\{C(b_{i-1,j-1}),C(b_{i-1,j}),C(b_{i-1,j+1})\} + |b_{i,j}|,& \text{de outra forma}
    \end{cases}
  \end{align*}

  No primeiro caso (base), considera-se que o custo de escalar para um bloco logo acima do chão é o custo do próprio bloco.

  No segundo caso (base), considera-se infinito o custo de chegar num bloco lateralmente fora da parede de escalada. Isso facilita o uso de índices horizontais sem a introdução de condicionais complicados no terceiro caso, para situações em que $j = 0$ e $j = m+1$.

  No terceiro caso, queremos encontrar o menor custo até $b_{i,j}$, considerando que só conseguimos alcançá-lo a partir do bloco diretamente abaixo dele ou dos blocos nas suas diagonais inferior direita e esquerda, $b_{i-1,j}$, $b_{i-1,j-1}$ e $b_{i-1,j+1}$, respectivamente.

  \section{Projeto de algoritmo}
  Primeiramente, pensei em transformar o problema em uma busca por menor caminho em grafo, criando um vértice por bloco e arestas para cada passagem possível entre blocos. Descartei essa solução após perceber que transformar a entrada dada em formato de matriz em um grafo seria muito complicado, podendo aumentar bastante a complexidade de memória.

  Ao analisar a recorrência do problema, percebi que, no $i$-ésimo nível, dependemos apenas dos custos do nível imediatamente abaixo para calcular os custos de se chegar em cada um de seus $j$ blocos. Assim cheguei na solução abaixo:

  \begin{quote}
    Vamos inicializar um vetor de custos $C$ de tamanho $m+2$ com infinito na primeira e última posições e os custos da primeira linha de blocos nas suas demais posições. Então, para cada nível da matriz de escalada a partir do segundo, de baixo para cima, vamos atualizar os valores de $C$ de modo que sua $(j+1)$-ésima posição contenha o custo de se chegar ao $j$-ésimo bloco daquele nível, incluindo o próprio bloco. Para calcular o custo total de um bloco, basta encontrar o mínimo custo entre os blocos a partir dos quais o alcançamos, ou seja, do bloco imediatamente abaixo e de suas diagonais inferiores, e somar com o custo do próprio bloco, armazenado na matriz de escalada. Ao final dos $n$ níveis, basta retornar o menor valor armazenado em $C$.
  \end{quote}

  O algoritmo desenvolvido usa programação dinâmica bottom-up, uma vez que soluciona-se os resultados dos subproblemas (blocos inferiores) antes de solucionar-se os problemas que deles dependem (blocos superiores).

  \section{Análise de complexidade}
  Quanto à complexidade de espaço, a única variável que depende das dimensões do problema é o vetor $C$, com $m+2$ posições. Logo, o algoritmo tem complexidade espacial $O(m)$.

  Quanto à complexidade de tempo, incializamos $C$ em $O(m)$ e, para cada um dos $n-1$ níveis da matriz a partir do segundo, percorremos cada um de seus $m$ elementos uma vez, fazendo um acesso para cada um de seus três vizinhos inferiores e calculando o mínimo entre esses valores. Então, percorremos $C$ uma última vez para encontrar seu menor valor, em $O(m)$. Por isso, o algoritmo tem complexidade temporal $O(nm)$.

  Se considerarmos que $m \in O(n)$, então o algoritmo tem complexidade espacial $O(n)$ e temporal $O(n^2)$.

\end{document}