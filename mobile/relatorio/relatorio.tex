\documentclass[12pt,a4paper]{article}

\usepackage[brazil]{babel}
\usepackage{amsmath}
% \usepackage{mathtools}

\title{Projeto de Algoritmo com Implementação nº 0}
\date{31 de outubro de 2020}
\author{Lucas Valente Viegas de Oliveira Paes - RA 220958}

\begin{document}
  \maketitle

  \section{Demonstração de corretude}
  Vamos demonstrar que o algoritmo funciona por indução.
  \paragraph{Hipótese de indução}
  Dado um móbile, sabemos calcular seu peso em ambas as suas extremidades, bem como saber se ambas estão em equilíbrio.
  \paragraph{Caso base} Mobile sem sub-móbiles atrelados às suas extremidades. Nesse caso, seu peso é $P_e + P_d$ e ele está em equilíbrio somente se $P_eD_e = P_dD_d$.
  \paragraph{Passo indutivo} Pela \emph{Hipótese de indução}, sabemos $P_e$ e $P_d$, além de saber se as extremidades do móbile estão equilibradas. Assim, determinamos que o seu peso é $P_{d} + P_{e}$ e ele está em equilíbrio somente se $P_eD_e = P_dD_d$.

  \section{Análise de complexidade no pior caso}
  O pior caso ocorre quando temos um móbile equilibrado cujo número de folhas, ou seja, de móbiles que não têm um submóbile atrelado a eles, é mínimo. Neste caso, o número de chamadas recursivas é maximizado, enquanto as ocorrências do caso base são minimizadas. 
  
  Uma instância dele pode ser representada por uma árvore com $n-1$ móbiles internos de um filho e apenas um móbile folha (sem nenhum filho). Para esse caso, a fórmula de recorrência é

  \begin{equation}\label{recorrencia}
    \begin{cases}
      T(1) = \Theta(1),& \text{se } n = 1\\
      T(n) = T(n-1) + \Theta(1),& \text{se } n > 1\\
    \end{cases}
  \end{equation}

  Vamos supor que $T(m) = m$ para $m < n$. Considerando $m = n-1$, $n > 1$ e substituindo em (\ref{recorrencia}), temos
  \begin{equation*}
    T(n) = T(m) + \Theta(1) = n-1 + \Theta(1)
  \end{equation*}
  No caso base ($m = 1$),
  \begin{equation*}
    T(1) = 1
  \end{equation*}
  Portanto, $T(n) \in \Theta(n)$.

  \section{Notas sobre a implementação}
  É sabido que a entrada do programa segue a ordem inversa de hierarquia da árvore, ou seja, que os nós pais são dados antes dos filhos. Sabe-se, ainda, que, para nós que têm filhos, estes aparecem logo em seguida, da esquerda para a direita. Em posse dessas informações, é possível desenhar uma uma abordagem não recursiva de travessia pela árvore, que começa a partir dos nós filhos e em seguida determina o balanceamento e peso do nó pai. Desse modo, evita-se erros de \emph{stack overflow} além da travessia de volta por toda a pilha de chamadas recursivas (\emph{call stack}) para retorno.

  A ideia por trás dessa implementação é inserir os móbiles lidos em uma pilha $P_m$ para acessá-los na ordem reversa, tendo à disposição uma pilha de pesos $P_p$. Ao desempilhar um móbile $M$ de $P_m$, checamos se ele tem filho esquerdo. Se sim, o peso deste filho está no topo de $P_p$. Fazemos o mesmo com o filho direito. Após isso, verificamos se $M$ está balanceado. Caso não, o móbile global está desbalanceado e podemos parar todo o processamento por aí. Caso sim, calculamos seu peso e adicionamos a $P_p$. Note que nunca consumiremos de $P_p$ vazia, pois sempre iniciaremos o processamento pelas folhas (casos base), já que elas são dadas por último na entrada do programa.

  Como a inserção e remoção nas pilhas ocorre em tempo constante ($O(1)$) e estamos calculando o peso e balanceamento de nós internos a partir de seus filhos, a demonstração de corretude e a análise de complexidade continuam valendo.


\end{document}